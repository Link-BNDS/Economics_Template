\documentclass[12pt]{article}
\title{AP Micro \& Macro Economics Monopoly Homework}
\author{Link Li \& Zizi Yang\pgfornament[width=0.7cm]{94}}
\date{\today}
\usepackage{graphicx}
\usepackage{tikz}
\usepackage{geometry}
\usepackage{qrcode}
\usepackage{pgfornament}
\usepackage{eso-pic}
\usepackage{tikzrput}
\usepackage{fontspec}
%\usepackage{boondox-cal}
%\usepackage{amsmath}

\setmainfont{texgyrepagella-regular.otf}[
BoldFont = texgyrepagella-bold.otf ,
ItalicFont = texgyrepagella-italic.otf ,
BoldItalicFont = texgyrepagella-bolditalic.otf ]

\setsansfont{texgyrepagella-regular.otf}[
BoldFont = texgyrepagella-bold.otf ,
ItalicFont = texgyrepagella-italic.otf ,
BoldItalicFont = texgyrepagella-bolditalic.otf ]

\setmonofont{texgyrepagella-regular.otf}[
BoldFont = texgyrepagella-bold.otf ,
ItalicFont = texgyrepagella-italic.otf ,
BoldItalicFont = texgyrepagella-bolditalic.otf ]

\usetikzlibrary{positioning, arrows.meta, chains, scopes, calc}
\newcommand\AtPageUpperRight[1]{\AtPageUpperLeft{%
 \put(\LenToUnit{\paperwidth},\LenToUnit{0\paperheight}){#1}%
 }}%
\newcommand\AtPageLowerRight[1]{\AtPageLowerLeft{%
 \put(\LenToUnit{\paperwidth},\LenToUnit{0\paperheight}){#1}%
 }}%
\AddToShipoutPictureBG{%
   \AtPageUpperLeft{\put(0,-25){\pgfornament[width=1.75cm]{35}}}
   \AtPageUpperRight{\put(-50,-25){\pgfornament[width=1.75cm,symmetry=v]{35}}}
   \AtPageLowerLeft{\put(0,25){\pgfornament[width=1.75cm,symmetry=h]{35}}}
   \AtPageLowerRight{\put(-50,25){\pgfornament[width=1.75cm,symmetry=c]{35}}}
}

\begin{document}
\maketitle

\section{}

Firms are always trying to maximize profit. This is the only "intension or incentive". Firms play two distinct and contrasted characters at two markets. One is a price taker, one is a price maker. Hence, their roles-or their ability to the market-is the only difference. In conclusion, it is the natures of markets that actually influence firm's behavior, not their incentives.


\section{}

Although monopolistic firm can control the price, but they still need to obey the law of demand. The max profit point is reached when MR=MC, not highest price.

\section{}

Yes. Because in PCM, firms can enter the market in long run if the economic profit exists. In contrast, although economic profit exists, other firms can't or don't willing to(natural monopoly) enter the market. So they can just watch it making economic profit. They can't do anything. 


\section{}


It is possible. It is less efficient than PCM. Zero profit equilibrium refers to P=ATC. But it is not allocative efficient, becuase MC does not equals to MB except there is a point that MC=MB=ATC.


\section{}

Education industry. Most of the student in China study GaoKao. Some of the students study international education, just like us. Students who study international education come from families which relatively have more wealth that enable them to pay the tuition. So they are willing to pay more on education.\\
Not only the schools, but also the "After-school institute". And "Agency".\\
BNDS international department's tuition is 100000 RMB per year without tests fees (TOEFL, AP, ACT, etc). The international education, from the result perspective, has smaller difference than the GAO KAO regards to price difference. (Not because the difference in result is small, but because the difference in price is much more than result.(100000 vs free))
\\
Another example is Music APP in China. They have a rule like "VIP listen for free-Non-VIP buy single". This rule separates the low and high willingness of people.
\\
Reqirement\\
1. To know each groups of people's willingness to pay, generally understanding is okay.. \\
2. Forbid the re-sale process. 


\section{}

In order to maximize profit, monopoly market will generate DWL, decreasing the total wealth and making the market not efficient. But in PCM, it is always allocative efficient.

\section{}
L\\
O\\
O\\
K\\
A\\
T\\
HERE












\end{document}